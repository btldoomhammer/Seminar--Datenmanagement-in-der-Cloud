\chapter{Algorithmen}
\label{sec:Algorithmen}

Verbesserungsideen: entnommen aus \cite{SALOMIE}

Einsatz dynamischer programmier-optimierung

einsatz von hilfs-prozessoren f�r pre-fetch data

\section{Cache Optimierung}
\label{sec:Algorithmen_Cache-Optimierung}


%\section{Minimalizing Cache Conflict}

\cite{LEE}


\section{Join Operationen}
\label{sec:Join}

Index-Join in Zukunft schneller als Hash-Joins \cite{KIM}

\subsection{Architecture Aware Hash-Join}
\label{sec:AA-Hash-Join}

Architecture Aware Hash-Join (AA-HJ) \cite{RASHID}

Daten werden im Speicher gehalten und kritische Daten auf Cache-Ebene werden verteilt. Verteilt in gleichen Teilen auf die Threads. Reduzierung der L2-Cache Miss-Rate(ca 80%).

\section{Sort Operationen}
\label{sec:Algorithmen_Sort}

Gro�es Problem/Leistungsverlust auf modernen CPUs durch Pipeline Abbr�che.

\subsection{AA-Sort}
\label{sec:Sort_2_AA-Sort}

AA-Sort \cite{INOUE} ist aufgeteilt in in-Core Sortierung und out-of-Core Sortierung. In-Core sortiert unter zuhilfename von Cache, out-of-Core macht vorsortierung in bl�cke, die in cache passen und macht ein merg mit den vom Core sortierten Bl�cken.

Kein Datenbank-Spezifischer Algorithmus aber ausnutzung mehrerer Cores und kann auch f�r DB-Anwendungen verwendet werden.

Sehr gut auf steigende Core-Zahlen skalierbar.

Ausnutzung der SIMD (single instruction multiple data) Anweisungen.

Schl�gt existierende Sortierungs-Algorithmen in Leistung.

O(n log(n))

Ablauf: (1)Teile Daten in Bl�cke die in den CPU Cache passen.
(2) Sortieren auf CPU mit in-Core Sortierungs-Algorithmus.
(3) Mergen der Sortierten Blocke mit dem out-of-Core Algorithmus

\subsubsection*{In-Core Sortierung}
\label{sec:In-Core_Sortierung}

Basiert auf combsort \cite{LACEY}, einer Verbesserung von Bubble-Sort.

\section{Evaluation}
\label{sec:Evaluation}

Allgemeine betrachtung und bewertung, welche Probleme und wie stark die Nutzung von Multicore ist und wieviel Leistung gewonnen wird. Aufwand/Nutzen

%\section{Aggregation}
%\label{sec:Aggregation}

%Shared Hash Tables vs multiple Hash Tables for each core ... \cite{CIESLEWICZ} (Vielleicht ein interessantes Paper ... )