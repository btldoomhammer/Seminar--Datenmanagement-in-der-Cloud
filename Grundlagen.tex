\chapter{Grundlagen}
\label{sec:Grundlagen}

Begriffe und Zusammenh�nge

Multicore - Multithread \cite{GARCIA}

SIMD(Single Instruction Multiple Data) Selbe Instruktion arbeitet mit mehreren Daten aus Pool.

interquery Parallelismus - Anzahl Queries die pareallel ausgef�hrt werden k�nnen \cite{HUBER}

interoperator Parallelismus - 

intraoperator  Parallelismus


\section{Multicore}
\label{sec:Multicore}

\subsection*{Vorteile}
\label{sec:Multicore_Vorteile}

Geteilter Cache:
Leistungsgewinn - Ein Thread bereitet Daten f�r einen anderen Thread vor
Potentiale f�r Leistungsverbesserung

\subsection*{Schwierigkeiten}
\label{sec:Multicore_Schwierigkeiten}

Problem der gr��eren Verwaltung bei mehreren Prozessoren. Parallelit�t erzeugt gr��ere Gefahr von Inkonsistenzen und Wartezust�nde, bei denen ein Thread auf den anderen warten muss, da beide die selben Daten bearbeiten wollen.

\section{Datenbank Operationen}
\label{sec:Operationen}

Insert, update, delete, join, sort kurz beschreiben mit bezug auf die Verwendung und optimierung durch Multi-Core Einsatz. Dient als einleitung/aussortierung der Operationen, die uninteressant sind in ihrer Betrachtung.

Hash-Join ist eine der wichtigsten Operationen in DBMS
