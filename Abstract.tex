\chapter*{Abstract}
\label{sec:Abstract}

In den letzten Jahren wird die Rechenleistung von Computern erhöht, indem Prozessoren mit mehreren Kernen ausgestattet werden. Datenbanksysteme sind allerdings zurzeit noch nicht in der Lage das vollständige Potenzial von Multicore-Prozessoren auszunutzen. In dieser Arbeit behandeln wir vier Algorithmen von verschiedenen Entwicklern, die Datenbanksysteme für Multicore-Architekturen optimieren sollen und vergleichen diese auf Nutzen und Aufwand. Dabei versuchen die Algorithmen sowohl das Caching der Prozessoren als auch Hash-Join und Sort-Operationen so wie die Zugriffspläne für die Nutzung der Mehrkernprozessoren zu verbessern. Es stellt sich dabei heraus, dass alle vier Algorithmen eine signifikante Leistungssteigerung ermöglichen, wobei mit mehr Implementierungsaufwand auch bessere Effekte erzielt werden. Die Algorithmen verbessern eindeutig die Performanz und könnten Einzug in die Datenbanksysteme erhalten.