\chapter{Einleitung}
\label{sec:Einleitung}

\section{Motivation}
\label{sec:Motivation}
Das Mooresche Gesetz aus den 60iger Jahren sagt aus, dass sich die Zahl der Transistoren auf integrierten Schaltungen alle 18 Monate verdoppelt. Dies wirkt sich auch analog auf die Rechenleistung aus. In den letzten Jahren wird die Erhöhung der Rechenleistung allerdings nicht mehr durch noch schnellere Prozessoren erzielt, sondern durch den Einsatz von mehreren Prozessorkernen, die dann eine Einheit bilden und (echte) parallele Verarbeitung ermöglichen.

Während bei der Verbesserung der Ein-Kern-Prozessoren die Datenbanksysteme ebenso direkt davon profitieren konnten, stellen sich bei Multicore-Prozessoren neue Herausforderungen an das Datenbanksystem. Um das komplette Potential der Multicore-Prozessoren ausreizen zu können, müssen Algorithmen angepasst und neu erfunden werden.

\section{Ziele der Arbeit}
\label{sec:ZieleDerArbeit}
In dieser Arbeit werden verschiedene Ansätze vorgestellt, wie man Datenbanken und deren Algorithmen modifizieren kann, um das volle Potenzial von Multicoreprozessoren auszunutzen. Dabei soll an verschiedenen Punkten angesetzt werden. 

[BILD Layout]

Optimierungen können hardwarenah auf Cache Ebene, sowie auf Anfrageoptimierungsebene stattfinden.

\section{Aufbau der Arbeit}
\label{sec:AufbauDerArbeit}
Im zweiten Kapitel werden die Grundlagen und Grundbegriffe über die Themen Multicoreprozessoren und Datenbank erläutert, damit jeder Leser den Ausführungen im Kapitel drei folgen kann. Im dritten Kapitel werden aktuelle Ansätze zur Optimierung der Datenbanken auf Multicoreprozessoren vorgestellt und anschließend auch bewertet. Dabei betrachten wir Optimierungen im Cache und bei Join- und Sort-Algorithmen. Das letzte Kapitel enthält eine Zusammenfassung unserer Erkenntnisse und gibt noch einen weiteren Ausblick auf weitere Ansätze in verwandten Themengebieten.